\documentclass[14pt]{extarticle}

\usepackage[a4paper,left=18mm,right=18mm,top=20mm,bottom=18mm]{geometry}
\usepackage[italian]{babel}
 
\usepackage{titling}
\usepackage{graphicx}
\usepackage{subcaption}
\usepackage{float}
 
\title{Formule che vengono utilizzate per la presentazione}
\date{\today}    
 
\begin{document}
\maketitle
\section{Formula completa per i ricavi}
$\sum_{i=1} ^{n} Mix_{i}\cdot VolumeTotale \cdot $$\frac{PrezzoUnitario_{i}}{TassoCambioMedio_{i}}$
\bigskip
\\$\frac{Quantity_{i}}{VolumeTotale}$ as Mix\\
\bigskip
\\VolumeTotale
\bigskip
\\$\frac{TotaleLocale}{Quantity}$ as PrezzoUnitario

\section{Lavorazione}
$\sum_{i=1} ^{n} ImpiegoUnitario_i \cdot CostoOrario_i \cdot QuantitydiOutput_i$
\bigskip
\\$\frac{TempoRisorsa_{i}}{QuantitydiOutput_{i}}$ as $ImpiegoUnitario_{i}$
\bigskip
\bigskip
\\CostoOrario
\bigskip
\bigskip
\\QuantitydiOutput

\section{Materia prima}
$\sum_{i=1} ^{n}QuantityOutput_i \cdot ImpiegoPerPezzo_i\cdot CostoPerUnitaMis_i$
\bigskip
\\$\frac{QuantityMPImpiegata_{i}}{QuantitydiOutput_{i}}$ as $ImpiegoPerPezzo_{i}$
\bigskip
\bigskip
\\$\frac {ImportoCostoTOTALE}{QuantityMPImpiegata_{i}}$ as CostoPerUnitaMis
\bigskip
\bigskip
\\QuantitydiOutput

\end{document}
