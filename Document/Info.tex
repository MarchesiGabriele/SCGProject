\documentclass{article}

\usepackage[a4paper,left=18mm,right=18mm,top=20mm,bottom=18mm]{geometry}
\usepackage[italian]{babel}

\usepackage{titling}
\usepackage{graphicx}
\usepackage{subcaption}
\usepackage{float}

\title{Considerazioni sul dataset di SCG }
\author{Ianitchii Alin, Gabriele Marchesi, David Guzman Piedrahita e Marco Vinciguerra}
\date{\today}    

\begin{document}
\maketitle
\begin{enumerate}
    \item \textbf{Tassi di cambio} $\rightarrow$ contiene i tassi di cambio sia a BUDGET che a CONSUNTIVO
    \item \textbf{Impiego orario risorse} $\rightarrow$ contiene articoli (colonna nr articolo) non unique. 
    \\Considerazione = ordine di produzione $\rightarrow$ prima c’è budget e poi consuntivo, sempre. Il controllo qualità viene eseguito sempre dopo nell’ordine di produzione. ART0000128 ha solo controllo di qualità.
    Controllo qualità ha sempre tempo di risorsa nullo. Fresatura ha quantità di output = 0.
    \item \textbf{Vendite} $\rightarrow$ la colonna Nr.Origine corrisponde all’id del cliente.
    \item \textbf{Consumi} $\rightarrow$ consumo di materia prima. Nr.documento $\rightarrow$ si riconduce all’ordine di produzione.
    \\Possibile camino di join $\rightarrow$ doppio join con la tabella impiego orario risorse.
    \item \textbf{Costo orario risorse} $\rightarrow$ Contiene il codice della risorsa e il costo orario della risorsa.
    \item \textbf{Clienti} $\rightarrow$ c’è il codice cliente e la valuta.
\end{enumerate}
\textbf{N.B:} Non c’è corrispondenza biunivoca tra budget e consuntivo.
\end{document}
